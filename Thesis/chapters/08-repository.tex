\chapter{Repozytorium}
\label{Chapter:Repository}

\noindent Repozytorium projektu zawiera wszystkie pliki, które zostały stworzone oraz z których skorzystano w ramach przygotowania pracy magisterskiej. Dostęp do repozytorium możliwy jest poprzez link \url{https://github.com/BartekPrz/DataStreamAnalysis}.

\noindent Struktura projektu prezentuje się następująco:

\begin{itemize}
    \item \textit{Algorithms} - folder zawierający kody źródłowe algorytmów napisanych w języku Java, które zostały przedstawione w niniejszej pracy
    \item \textit{Results} - folder zawierający wyniki poszczególnych algorytmów na określonych strumieniach danych oraz wyniki testów statystycznych
    \item \textit{Plots} - folder zawierający wykresy liniowe wyników algorytmów dla określonych strumieni danych
    \item \textit{Scripts} - folder zawierający kody źródłowe potrzebne do uruchomienia eksperymentów oraz stworzenia wykresów
    \item \textit{Thesis} - folder zawierający pliki źródłowe .tex napisanej pracy dyplomowej
    \item \textit{OtherAlgorithms} - folder zawierający kody źródłowe algorytmów napisanych w języku Java, które ostatecznie nie zostały przedstawione w niniejszej rozprawie
    \item \textit{OtherResults} - folder zawierający wyniki na określonych strumieniach danych dla algorytmów, które ostatecznie nie zostały przedstawione w pracy
\end{itemize}